% Chapter 1:
% ----------
\section{Introduction}\label{chap:chap_1}


% Section 1:
%-----------
\subsection{Motivation}
Transportation nowadays has acquired a new round of development with the introduction of autonomous driving technologies based on machine learning (ML), neural networks (NN), and artificial intelligence (AI). This changed how we think about safety, efficiency and accessibility of transport. The key challenge of a self-driving vehicle is the number of decisions, required for an algorithm to be taken in real-time which will cause endangering consequences for drivers, passengers and other traffic participants. Reinforcement Learning (RL) is a promising approach to tackle this challenge because it allows systems to learn by trying different actions and improving over time. However, to train RL sufficiently for real-world driving it needs a lot of time and big data sets for the learning process.\\
The major drawback of RL is the lack of adaptivity of models to new situations. This requires an additional learning process, which is time-consuming and inefficient. This is where Transfer Learning (TL) can help. TL enables the ability of a system to take what it has learned in one situation and then apply gathered knowledge to a new one, with less effort and time taken. Emphasizing the focus of TL to be trained on critical states of the system - the moments where making the right decision matters for safety and performance -  can lead to an exclusive ability to learn faster and perform better under real-world conditions.\\
This thesis explores the approach to make RL better at knowledge transfer in the field of autonomous driving by applying critical situations to the models' training process. The aim is to improve the performance of models, broadening the variety of challenges that could be handled, and ensuring the effectiveness of learning. By addressing these issues, this research could contribute to the further development of autonomous driving systems by making them more practical and reliable.\\


% Section 2:
%-----------
\subsection{Objective}




% Section 3:
%-----------
\subsection{Scope}



% Section 4:
%-----------
\subsection{Commonly used terms}

\begin{enumerate}
	\item \textbf{Commonly used term 1}
	
	\item \textbf{Commonly used term 2}
	
	\item \textbf{Commonly used term 3}
\end{enumerate}


% Section 5:
%-----------
\subsection{Thesis-outline}

\begin{enumerate}[label=\Roman*]
	\item {\color{red} \textbf{Chapter \ref{chap:chap_1}}}: Introduction\\
	
	\item {\color{red} \textbf{Chapter \ref{chap:chap_2}}}: Literature Survey\\
	
	\item {\color{red} \textbf{Chapter \ref{chap:chap_3}}}: Method\\
	 
	\item {\color{red} \textbf{Chapter \ref{chap:chap_4}}}: Experiments, Results and Discussion\\
	
	\item {\color{red} \textbf{Chapter \ref{chap:chap_5}}}: Conclusion\\
    
\end{enumerate}
